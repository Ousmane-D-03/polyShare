\documentclass[a4paper,11pt]{article}
\usepackage[utf8]{inputenc}
\usepackage[T1]{fontenc}
\usepackage[french]{babel}
\usepackage{geometry}
\usepackage{enumitem}
\usepackage{xcolor}
\usepackage{titlesec}
\usepackage{hyperref}
\usepackage{tikz}
\usepackage{fontawesome5}

\geometry{margin=2.5cm}

% Couleurs
\definecolor{primary}{RGB}{0, 102, 204}
\definecolor{secondary}{RGB}{255, 140, 0}
\definecolor{success}{RGB}{40, 167, 69}
\definecolor{danger}{RGB}{220, 53, 69}

% Titres
\titleformat{\section}{\Large\bfseries\color{primary}}{\thesection}{1em}{}[\titlerule]
\titleformat{\subsection}{\large\bfseries\color{secondary}}{\thesubsection}{1em}{}

% En-tête
\title{\Huge\bfseries PolyShare \\ \Large Plateforme de partage de polycopiés offline-first}
\author{Projet étudiant - Sénégal}
\date{\today}

\begin{document}

\maketitle

\begin{center}
\large\textit{``Télécharge une fois, lis toute la semaine''}
\end{center}

\vspace{1cm}

\tableofcontents
\newpage

%-------------------------------------------------------------------
\section{Vision du projet}

\subsection{Problématique}
Les étudiants au Sénégal (UCAD, UGB, etc.) rencontrent plusieurs obstacles majeurs :
\begin{itemize}[label=\textcolor{danger}{\faTimesCircle}]
    \item Polycopiés hors de prix (2000-5000 FCFA par poly)
    \item Connexion internet faible et instable
    \item Bibliothèques surchargées et sous-équipées
    \item Diffusion anarchique via WhatsApp
    \item Pas de plateforme centralisée
\end{itemize}

\subsection{Solution proposée}
\textbf{PolyShare} : une PWA (Progressive Web App) permettant aux étudiants de :
\begin{itemize}[label=\textcolor{success}{\faCheckCircle}]
    \item Partager gratuitement leurs polycopiés
    \item Télécharger des polys pour consultation offline
    \item Organiser leur bibliothèque par fac/filière/matière
    \item Fonctionner même sans connexion internet
\end{itemize}

\subsection{Avantages compétitifs}
\begin{enumerate}
    \item \textbf{Offline-first} : fonctionne sans connexion constante
    \item \textbf{Ultra-léger} : optimisé pour connexions faibles
    \item \textbf{Gratuit} : pas de barrière financière
    \item \textbf{PWA} : pas besoin du Play Store, installation directe
    \item \textbf{Partage viral} : via QR code et Bluetooth (future)
\end{enumerate}

%-------------------------------------------------------------------
\newpage
\section{Architecture technique}

\subsection{Stack technologique}

\subsubsection{Frontend (PWA)}
\begin{itemize}
    \item \textbf{Framework} : React 18+ avec Vite
    \item \textbf{Styling} : TailwindCSS
    \item \textbf{Offline} : Service Worker + Workbox
    \item \textbf{Stockage local} : IndexedDB (via Dexie.js)
    \item \textbf{Lecture PDF} : PDF.js ou React-PDF
    \item \textbf{UI Components} : shadcn/ui ou Headless UI
\end{itemize}

\subsubsection{Backend (API REST)}
\begin{itemize}
    \item \textbf{Runtime} : Node.js 20+ avec Express.js
    \item \textbf{Base de données} : PostgreSQL (métadonnées)
    \item \textbf{Stockage fichiers} : AWS S3 ou Cloudinary
    \item \textbf{Cache} : Redis (optionnel phase 1)
    \item \textbf{Authentification} : JWT
\end{itemize}

\subsubsection{Hébergement}
\begin{itemize}
    \item \textbf{Frontend} : Vercel ou Netlify (gratuit)
    \item \textbf{Backend} : Railway, Render ou Fly.io (gratuit au début)
    \item \textbf{Base de données} : Supabase ou Neon (PostgreSQL gratuit)
    \item \textbf{CDN} : Cloudflare (gratuit)
\end{itemize}

\subsection{Fonctionnalités principales}

\subsubsection{Phase MVP (Minimum Viable Product)}
\begin{enumerate}
    \item Inscription/connexion simple (email + mot de passe)
    \item Navigation par fac/filière/matière/semestre
    \item Upload de polycopiés (PDF uniquement)
    \item Téléchargement pour lecture offline
    \item Bibliothèque personnelle (stockage local)
    \item Recherche basique
    \item Système de karma (upload = points pour télécharger)
\end{enumerate}

\subsubsection{Phase 2 (post-MVP)}
\begin{itemize}
    \item Annotations sur PDF
    \item Partage Bluetooth/Wi-Fi Direct
    \item Mode "Pack de la semaine"
    \item Signalement de contenu inapproprié
    \item Statistiques personnelles
    \item Notifications push
\end{itemize}

\subsubsection{Phase 3 (monétisation)}
\begin{itemize}
    \item Intégration Orange Money / Wave
    \item Abonnement Premium
    \item Anciens sujets d'examens
    \item Stockage cloud illimité
\end{itemize}

%-------------------------------------------------------------------
\newpage
\section{Planification détaillée}

\subsection{Phase 0 : Préparation (Semaine 1-2)}

\subsubsection{Validation du besoin}
\begin{itemize}[label=\faSquare]
    \item Interviewer 15-20 étudiants sénégalais
    \item Analyser la concurrence existante
    \item Définir les personas utilisateurs
    \item Valider les hypothèses clés
    \item Créer un document de spécifications fonctionnelles
\end{itemize}

\subsubsection{Design UX/UI}
\begin{itemize}[label=\faSquare]
    \item Wireframes basse fidélité (papier/Excalidraw)
    \item Maquettes haute fidélité (Figma)
    \item Prototype cliquable
    \item Tests utilisateurs (5 personnes minimum)
    \item Itération sur le design
\end{itemize}

\subsubsection{Modélisation}
\begin{itemize}[label=\faSquare]
    \item Diagramme de cas d'utilisation (UML)
    \item Diagramme de classes
    \item Modèle de données (MCD/MLD)
    \item Diagramme de séquence (actions principales)
    \item Architecture système
\end{itemize}

\subsubsection{Setup environnement}
\begin{itemize}[label=\faSquare]
    \item Créer le repo GitHub
    \item Initialiser le projet frontend (Vite + React)
    \item Initialiser le projet backend (Express)
    \item Setup ESLint + Prettier
    \item Configurer le CI/CD de base
\end{itemize}

\subsection{Phase 1 : Développement MVP (Semaine 3-8)}

\subsubsection{Semaine 3-4 : Backend fondations}
\begin{itemize}[label=\faSquare]
    \item Setup base de données PostgreSQL
    \item Modèles : User, Document, Category, University, Faculty
    \item API Auth : register, login, logout, refresh token
    \item API Documents : upload, list, get by ID, delete
    \item Middleware : auth, upload fichiers, validation
    \item Tests unitaires (Jest)
\end{itemize}

\subsubsection{Semaine 5-6 : Frontend core}
\begin{itemize}[label=\faSquare]
    \item Setup PWA (manifest.json, service worker)
    \item Pages : Home, Login, Register, Browse, Document Detail
    \item Composants réutilisables : Card, Button, Input, Modal
    \item Intégration API auth
    \item Intégration API documents (liste, détail)
    \item Système de routing (React Router)
\end{itemize}

\subsubsection{Semaine 7 : Fonctionnalités offline}
\begin{itemize}[label=\faSquare]
    \item Service Worker avec Workbox
    \item Cache stratégies (Cache First pour PDFs)
    \item IndexedDB setup avec Dexie.js
    \item Download manager (téléchargement + stockage local)
    \item Bibliothèque offline (lecture depuis IndexedDB)
    \item Sync queue pour uploads en attente
\end{itemize}

\subsubsection{Semaine 8 : Upload et recherche}
\begin{itemize}[label=\faSquare]
    \item Formulaire upload avec validation
    \item Compression PDF côté client (optionnel)
    \item Upload avec progression
    \item Système de recherche (fac/filière/matière/keyword)
    \item Filtres avancés
    \item Système de karma (upload = +10 points, download = -1 point)
\end{itemize}

\subsection{Phase 2 : Tests et optimisation (Semaine 9-10)}

\subsubsection{Tests}
\begin{itemize}[label=\faSquare]
    \item Tests end-to-end (Playwright ou Cypress)
    \item Tests de charge (combien d'utilisateurs simultanés ?)
    \item Tests offline complets
    \item Tests sur connexion lente (throttling)
    \item Tests multi-devices (Android, iOS, Desktop)
\end{itemize}

\subsubsection{Optimisations}
\begin{itemize}[label=\faSquare]
    \item Optimisation taille bundle (code splitting)
    \item Lazy loading des composants
    \item Compression images et assets
    \item Optimisation requêtes SQL (indexes)
    \item Mise en cache API (Redis ou in-memory)
\end{itemize}

\subsubsection{Sécurité}
\begin{itemize}[label=\faSquare]
    \item Rate limiting (brute force protection)
    \item Validation robuste des uploads (type MIME, taille)
    \item Sanitization des inputs
    \item Protection CSRF
    \item Headers sécurisés (CORS, CSP)
    \item Scan vulnérabilités (npm audit)
\end{itemize}

\subsection{Phase 3 : Déploiement et lancement (Semaine 11-12)}

\subsubsection{Déploiement}
\begin{itemize}[label=\faSquare]
    \item Deploy backend sur Railway/Render
    \item Deploy frontend sur Vercel/Netlify
    \item Setup base de données production (Supabase)
    \item Configuration CDN Cloudflare
    \item Setup monitoring (Sentry pour errors)
    \item Setup analytics (Plausible ou Umami)
\end{itemize}

\subsubsection{Bêta test}
\begin{itemize}[label=\faSquare]
    \item Recruter 30 étudiants testeurs (UCAD priorité)
    \item Créer groupe WhatsApp support
    \item Onboarding des testeurs
    \item Collecte feedback quotidien
    \item Itérations rapides (hotfix en 24h)
\end{itemize}

\subsubsection{Marketing pré-lancement}
\begin{itemize}[label=\faSquare]
    \item Page landing explicative
    \item Tutoriel vidéo (3 min max)
    \item Affiches à imprimer (QR code)
    \item Posts réseaux sociaux (templates)
    \item Contacter délégués de classe
    \item Préparer pitch pour associations étudiantes
\end{itemize}

\subsection{Phase 4 : Lancement public (Semaine 13+)}

\subsubsection{Semaine 13 : Lancement soft}
\begin{itemize}[label=\faSquare]
    \item Annonce dans 5 groupes Facebook étudiants
    \item Partage dans 10 groupes WhatsApp
    \item Affiches dans 3 facultés UCAD
    \item Post LinkedIn ciblé Sénégal
    \item Objectif : 100 utilisateurs semaine 1
\end{itemize}

\subsubsection{Semaine 14-16 : Croissance}
\begin{itemize}[label=\faSquare]
    \item Partenariat avec 2-3 associations étudiantes
    \item Campagne "Partage ton poly, gagne un abonnement"
    \item Premier rapport de métriques
    \item Identifier les power users (ambassadeurs)
    \item Expansion vers UGB si succès UCAD
\end{itemize}

%-------------------------------------------------------------------
\newpage
\section{Modélisation UML}

\subsection{Cas d'utilisation principaux}

\subsubsection{Acteurs}
\begin{itemize}
    \item \textbf{Étudiant non connecté} : peut naviguer, voir les polys
    \item \textbf{Étudiant connecté} : peut télécharger, uploader
    \item \textbf{Modérateur} (future) : valide les uploads
    \item \textbf{Système} : gestion offline, sync
\end{itemize}

\subsubsection{Cas d'utilisation}
\begin{enumerate}
    \item S'inscrire / Se connecter
    \item Naviguer par fac/filière/matière
    \item Rechercher un polycopié
    \item Télécharger un poly (pour offline)
    \item Uploader un polycopié
    \item Consulter sa bibliothèque offline
    \item Gagner/dépenser des points karma
    \item Signaler un contenu
\end{enumerate}

\subsection{Modèle de données}

\subsubsection{Entités principales}
\begin{verbatim}
User {
    id: UUID (PK)
    email: String (unique)
    password_hash: String
    username: String
    karma_points: Integer (default: 10)
    role: Enum ['student', 'moderator', 'admin']
    university_id: UUID (FK)
    created_at: Timestamp
    updated_at: Timestamp
}

University {
    id: UUID (PK)
    name: String (ex: "UCAD")
    country: String (default: "Sénégal")
    city: String
}

Faculty {
    id: UUID (PK)
    name: String (ex: "Sciences et Techniques")
    university_id: UUID (FK)
}

Major {
    id: UUID (PK)
    name: String (ex: "Informatique")
    faculty_id: UUID (FK)
}

Course {
    id: UUID (PK)
    name: String (ex: "Algorithmique")
    code: String (ex: "INFO301")
    major_id: UUID (FK)
    semester: Integer
}

Document {
    id: UUID (PK)
    title: String
    description: Text
    course_id: UUID (FK)
    uploaded_by: UUID (FK -> User)
    file_url: String (S3/Cloudinary)
    file_size: Integer (bytes)
    file_hash: String (SHA256, évite doublons)
    downloads_count: Integer (default: 0)
    status: Enum ['pending', 'approved', 'rejected']
    created_at: Timestamp
    updated_at: Timestamp
}

Download {
    id: UUID (PK)
    user_id: UUID (FK)
    document_id: UUID (FK)
    downloaded_at: Timestamp
}
\end{verbatim}

\subsubsection{Relations}
\begin{itemize}
    \item Un User appartient à une University (N-1)
    \item Une University a plusieurs Faculties (1-N)
    \item Une Faculty a plusieurs Majors (1-N)
    \item Un Major a plusieurs Courses (1-N)
    \item Un Course a plusieurs Documents (1-N)
    \item Un User peut uploader plusieurs Documents (1-N)
    \item Un User peut télécharger plusieurs Documents (N-N via Download)
\end{itemize}

%-------------------------------------------------------------------
\newpage
\section{Métriques de succès}

\subsection{Objectifs MVP (3 mois)}
\begin{itemize}
    \item \textbf{500 utilisateurs inscrits} (UCAD uniquement)
    \item \textbf{100 polycopiés uploadés}
    \item \textbf{2000 téléchargements} au total
    \item \textbf{40\% de rétention} semaine 2
    \item \textbf{20\% d'utilisateurs actifs hebdomadaires}
\end{itemize}

\subsection{KPIs à suivre}
\begin{enumerate}
    \item \textbf{Acquisition} : nouveaux inscrits / semaine
    \item \textbf{Activation} : \% qui téléchargent au moins 1 poly
    \item \textbf{Rétention} : utilisateurs actifs semaine N / semaine N-1
    \item \textbf{Engagement} : temps moyen passé offline sur l'app
    \item \textbf{Viralité} : \% d'utilisateurs qui invitent d'autres
    \item \textbf{Contribution} : ratio uploads / downloads
\end{enumerate}

%-------------------------------------------------------------------
\section{Budget et ressources}

\subsection{Coûts estimés (6 premiers mois)}
\begin{itemize}
    \item \textbf{Hébergement} : 0-20\$ / mois (gratuit au début)
    \item \textbf{Stockage S3} : \textasciitilde10\$ / mois (500 polys)
    \item \textbf{Nom de domaine} : 12\$ / an (.sn ou .com)
    \item \textbf{Outils} : Figma (gratuit), VS Code (gratuit)
    \item \textbf{Marketing} : 50\$ (impression affiches)
    \item \textbf{TOTAL} : \textasciitilde100\$ les 6 premiers mois
\end{itemize}

\subsection{Temps requis}
\begin{itemize}
    \item \textbf{Développement MVP} : 10-15h / semaine × 10 semaines = 100-150h
    \item \textbf{Tests et déploiement} : 20h
    \item \textbf{Marketing et support} : 5h / semaine continu
\end{itemize}

%-------------------------------------------------------------------
\section{Risques et mitigations}

\subsection{Risques techniques}
\begin{enumerate}
    \item \textbf{Stockage local limité sur téléphones}
    \begin{itemize}
        \item[$\rightarrow$] Mitigation : compression PDFs, auto-nettoyage
    \end{itemize}
    
    \item \textbf{Service Worker complexe}
    \begin{itemize}
        \item[$\rightarrow$] Mitigation : utiliser Workbox (simplifie)
    \end{itemize}
    
    \item \textbf{Upload de contenu inapproprié}
    \begin{itemize}
        \item[$\rightarrow$] Mitigation : modération + signalement
    \end{itemize}
\end{enumerate}

\subsection{Risques business}
\begin{enumerate}
    \item \textbf{Faible adoption initiale}
    \begin{itemize}
        \item[$\rightarrow$] Mitigation : focus sur 1 fac, ambassadeurs
    \end{itemize}
    
    \item \textbf{Problèmes de droits d'auteur}
    \begin{itemize}
        \item[$\rightarrow$] Mitigation : CGU claires, disclaimer
    \end{itemize}
    
    \item \textbf{Concurrence émergente}
    \begin{itemize}
        \item[$\rightarrow$] Mitigation : vitesse d'exécution, communauté
    \end{itemize}
\end{enumerate}

%-------------------------------------------------------------------
\section{Prochaines étapes immédiates}

\subsection{Cette semaine}
\begin{enumerate}
    \item \fbox{Valider l'idée} : interviewer 10 étudiants
    \item \fbox{Setup repos} : GitHub frontend + backend
    \item \fbox{Wireframes} : esquisser les 5 écrans principaux
    \item \fbox{Modélisation} : diagramme de classes basique
\end{enumerate}

\subsection{Semaine prochaine}
\begin{enumerate}
    \item \fbox{Maquettes Figma} : design système complet
    \item \fbox{Setup projet} : Vite + React + Express
    \item \fbox{Premier endpoint} : API register/login
    \item \fbox{Première page} : Home avec navigation
\end{enumerate}

%-------------------------------------------------------------------
\vspace{2cm}
\begin{center}
\Large\textbf{Let's build something that matters! 🚀}

\vspace{0.5cm}
\normalsize
Contact : [ton email]\\
GitHub : [ton username]\\
Localisation : France → Sénégal
\end{center}

\end{document}
